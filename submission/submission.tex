\documentclass{article}
\author{Alex Hiller}
\title{Fields \& Waves -- Write Up}
% Type-setting
\setlength{\parindent}{0cm}
\setlength{\parskip}{0.125cm}
%\pagenumbering{gobble}
\usepackage[margin=2.5cm]{geometry} % Formatting
\usepackage{amsmath}      % Mathematics
\usepackage{amssymb}      % Mathematics
\usepackage{listings}     % Listings
\usepackage{esint}        % Mathematics
\usepackage{color}        % Listings
\usepackage{courier}      % Listings
\usepackage{circuitikz}   % Circuits
\usepackage{titlesec}     % Section Formatting
\usepackage{stmaryrd}     % \mapsfrom arrow. 
\usepackage{mathtools}    % \coloneqq
\usepackage{svg}
\usepackage{multicol}
\columnsep=75pt
\usepackage{blindtext}
\setsvg{inkscape=inkscape -z -D}
% \input{/home/polluticorn/GitHub/texTemplates/texMacros}
% Section formatting
\titleformat{\section}{\huge \bfseries}{}{0em}{}[]
\titleformat{\subsection}{\Large \bfseries}{}{0em}{}
\titleformat{\subsubsection}{\bfseries}{}{0em}{}

%%%%%%%%%%%%%%%%%%%%%%%%%%%%%%%%%%%%%%%%%%%%%%%%%%%%%%%%%%%%%%%%%%%%%%%%%%%%%%%%
\begin{document}
\maketitle 
\tableofcontents
\clearpage

%%%%%%%%%%%%%%%%%%%%
\section{Problem Sheet 1} 

\subsection{Griffiths, Chapter 3, Problem 54}

\subsubsection{Note} 
Positioning used is that present in the original question, \textbf{not} placement
of the pipe's corner on $z=0$.

\subsubsection{Working} 
Using separation of variables and the fact that the voltage function obeys Laplace's Equation, we get:

\[%
    V(x,y) 
    =
    (A e^{ky}+ B e^{-ky})
    \left(C \sin\left(kx\right) 
    +
    D \cos\left(kx\right) \right)
\]%

\begin{table}[!htbp]
    \centering
    \begin{tabular}{|c|c|} \hline
        & Boundary Values  \\ \hline
        (i) & $ V(x,0) $ = 0      \\ \hline
        (ii) & $ V(x,a) $ = $V_0$  \\ \hline
        (iii) & $ V(b,y) $ = 0 \\ \hline
        (iv) & $ V(-b,y)$ = 0 \\ \hline
    \end{tabular}
\end{table}

First thing is that there's a common approach of "it's symmetric across the
x-axis" and so the coefficient $C$ for $C \cdot \sin\left(kx\right) $

We'll abuse that convention to simplify the calculation.

Now we have:
\[%
    V(x,y) 
    =
    (A e^{ky}+ B e^{-ky})
    D \cos\left(kx\right) 
\]%

Using limit (i): 


\[%
    V(x,0) 
    =
    (A+B)D  \cos\left(kx\right) 
    =
    0
\]%

If we assume $D \neq 0$ :

\[%
     \therefore  \ 
    A = -B
\]%

We now have: 
\[%
    (-Be^{ky} + B e^{-ky}) D \cos\left(kx\right) 
\]%

Let's simplify the $y$-function.
 
\[%
    -B(e^{ky}- e^{-ky})D \cos\left(kx\right) 
\]%


\[%
    \frac{B'}{2}
    =
    -B
\]%

Meaning our $y$-function is:

\[%
    B'\sinh (ky)
\]%

And our voltage function is:


\[%
    B'\sinh (ky)
    D \cos\left(kx\right) 
\]%

Let's combine our coefficients:

\[%
    A' 
    =
    B'D 
    \qquad
    \Rightarrow 
    \qquad
    V(x,y)
    = 
    A' \sinh(ky) \cos\left(kx\right) 
\]%

Using boundary (iii):
\[%
    V(b,y)
    =
    0
    \qquad
    \Rightarrow 
    \qquad
    A' \sinh(ky) \cos\left(kb\right) 
    =
    0 
    \qquad
    \Rightarrow 
    \qquad
    \cos\left(kb\right) 
    =
    0
\]%

So we actually have infinite solutions for $kb$.


\[%
    kb 
    =
    \sum_{n=1}^{\infty} \pm \frac{n {\pi}}{2}
\]%

But $b$ is a constant value of a certain measurement of our pipe.
 
\[%
     \therefore \ k = \sum_{n=1}^{\infty} \frac{(2n-1)\pi}{2b}
\]%

So now we have a many different natural frequencies that satisfy the conditions.

Let's say that the $n$-th term of $k$ is $\alpha_n$.

\[%
    V(x,y) 
    =
    \sum_{n=1}^{\infty} A'_{n} \sinh\left(\alpha_n y\right) \cos\left(\alpha_n x\right) 
\]%

Using the $V_0$ boundary condition:


\[%
    V_0 
    =
    \sum_{n=1}^{\infty} A'_{n} \sinh\left(\alpha_n y\right) \cos\left(\alpha_n x\right) 
\]%

Multiplying $\cos (\alpha_n x)$ on both sides of the equation and integrating
from $b$ to $-b$ gives:

% When the frequencies are the same, it evaluates to dirac function? Picks it out
% of the series.

\[%
    V_0 \cos\left(\alpha_{n'} x\right)
    =
    \sum_{n=1}^{\infty} A'_{n} \sinh\left(\alpha_n y\right) \cos\left(\alpha_{n'} x\right) \cos\left(\alpha_n x\right) 
\]%

\[%
    V_0 
    \int^{b}_{-b} 
    \cos\left(\alpha_{n'} x\right)
    dx 
    =
    \sum_{n=1}^{\infty} A'_{n} \sinh\left(\alpha_n y\right) 
    \int^{b}_{-b} 
    \cos\left(\alpha_{n'} x\right) \cos\left(\alpha_n x\right) 
    dx 
\]%


\[%
    \because \ \
    \sum_{n=1}^{\infty} A'_{n} \sinh\left(\alpha_n y\right) 
    \int^{b}_{-b} 
    \cos\left(\frac{(2n-1)\pi}{2b}x\right) 
    \cos\left(\frac{(2n-1)\pi}{2b}x\right)  \ dx 
    =
    A'_{n} \sinh\left(\alpha_n y\right) 
    b\left(1-\frac{\sin(2n {\pi})}{(2n-1){\pi}}\right)
    =
    A'_{n} \sinh\left(\alpha_n y\right) 
    b
\]%


\[%
     \therefore 
     A'_{n} = 
     \frac{2V_0}{b}  
     \frac{\sin(\alpha_n b)}{\alpha_n \sinh\left(\alpha_n\right) a}
     \cos\left(\alpha_n x\right) 
\]%


\[%
    V(x,y) 
    =
    \sum_{n=1}^{\infty} 
    \frac{2V_0}{b}  
    \frac{\sin(\alpha_n b)\sinh(\alpha_n y)}{\alpha_n \sinh\left(\alpha_n\right) a}
    \cos\left(\alpha_n x\right) 
\]%


\clearpage
\subsection{Griffiths, Chapter 3, Problem 40} 



%%%%%%%%%%%%%%%%%%%%
\clearpage
\section{Problem Sheet 2} 


\subsection{Question 2} 
\[%
    \nabla  \times \mathbf{F} = 0
\]%


%%%%%%%%%%%%%%%%%%%%
\section{FAW Notes - Chapter 1} 

%%%%%%%%%%%%%%%%%%%%
\section{FAW Notes - Chapter 2} 

%%%%%%%%%%%%%%%%%%%%
\section{Other Topics} 

\subsection{Fourier Series}
\subsection{Differential Equations} 
\subsection{Flux} 
\subsection{Symmetry} 
\subsection{Proof of Cauchy-Riemann Equations} 
\subsection{Finite Fourier Sine Transform} 
\subsection{On Learning} 




\end{document}

